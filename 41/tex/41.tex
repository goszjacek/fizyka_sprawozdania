\documentclass[11pt]{article}

\usepackage[T1]{fontenc}
\usepackage[polish]{babel}
\usepackage[utf8]{inputenc}
\usepackage{lmodern}
\usepackage{amsfonts}
\usepackage{enumerate}
\usepackage{graphicx}
\usepackage{float}
\usepackage[margin=1in]{geometry}

\selectlanguage{polish}
\graphicspath{{./images/}}

\begin{document}
\title{Sprawozdanie 123}
\author{Jacek Gosztyła, Antoni Mleczko}
\maketitle
\section{Cel ćwiczenia} 
Celem ćwiczenia było zapoznanie się ze sposobem wyznaczania poziomej składowej magnetycznej\
magnetycznego pola ziemskiego przy użyciu przyrządu nazywanego busolą stycznych,\
a także zaobserwowanie podstawowych praw fizycznych wiążących prąd elektryczny z indukcją magnetyczną.
\section{Wstęp teoretyczny}
Busola stycznych to przedmiot, za pomocą którego jesteśmy w stanie wyznaczyć składową horyzontalną\
pola magnetycznego Ziemi. Aby dokonać tego pomiaru, pomoże nam prawo Biota-Savarta, które mówi, że\
$$ d\mathbf B = \frac{\mu_{0}I}{4\pi}\frac{d\mathbf{l} \times \mathbf{r}}{r^{3}} $$
Z powyższego prawa jesteśmy w stanie wyliczyć pole dla środka cewki kołowej lub bardzo krótkiej zwojnicy złożonej z N zwojów - wynosi ono:
$$ B = \mu_{0} * \frac{NI}{2R} $$
gdzie: \\
N - liczba zwojów\\
R - promień cewki\\
I - natężenie prądu\\
\\
Jest to wzór na całkowite pole magnetyczne wytwarzane w środku przewodnika kołowego, jednak\
aby obliczyć składową horyzontalną pola magentycznego ziemi $B_{0}$ musimy posłużyć się faktem, że\
$$\frac{B}{B_{0}} = tg\alpha$$
a zatem
$$ B_{0} = \mu_{0} * \frac{NI}{2Rtg\alpha} $$
Powyższy wzór pozwoli nam wyliczyć składową horyzontalną Ziemskiego pola magnetycznego.
\section{Realizacja doświadczenia}
W realizacji doświadczenia posłużyliśmy się układem przedstawionym na Rys. w1. w załączonych materiałach.\
Dla danej ilości zwojów dokonywaliśmy czterech pomiarów, każdy dla innej wartości płynącego prądu i\
w konsekwencji innego kąta - z zakresu 25-60 stopni. Pomiary zostały dokonane dla ilości zwojów: 40, 36, 24, 16.
\subsection{Pomiary}
\textbf{Wstaw tu tabelkę z pomiarami i B0!! <3}\\
klasa amperomierza - $4mA$ (dla ustalonego zakresu $300mA$)\\
średnica cewki - $260mm$\\
niepewność pomiaru średnicy cewki - $3mm$\\
\\
Obliczone wartości średnie i błędy pomiaru znajdują się w załączonych materiałach.
Średnia wartość zmierzonego $B{0}$ wyniosła $B_{0} = 25.07\mu T$.\\
Wartośc niepewności rozszerzonej wyniosła $u_{c} = 1.25\mu T$.\\
Wartość tabelaryczna składowej horyzontalnej pola magnetycznego dla Krakowa wynosi: $19.86\mu T$ (odczyt\
z dnia 27.11.2018)\\ 
Uwzględniając niepewność rozszerzoną i przyjmując za współczynnik rozszerzenia $k = 2$, nasze zmierzone $B_{0}$\
zawiera się w przedziale od $22.57\mu T$ do $27.57 \mu T$.
\section{Wnioski}
Zmierzona przez nas składowa horyzontalna magnetycznego pola ziemskiego, po uwzględnieniu rozszerzenia o\
niepewność rozszerzoną, nie pokrywa się z wartością rzeczywistą. Spowodowane jest to niedokładnością urządzeń\
pomiarowych użytych w doświadczeniu, a także dosyć sporymi zakłóceniami pola magnetycznego dookoła\
w trakcie wykonywania doświadczenia (wszelkie urządzenie elektroniczne).\\
Co zaskakujące jednak, z obliczeń niepewności rozszerzonej wynika iż niepewność pomiaru kątu wychylenia igły\
ma najmniejszy wpływ na niepewność złożoną. Największy wpływ ma pomiar prądu na amperomierzu.\
Wynik ten został zaobserwowany ponieważ amperomierz był analogowy i jego klasa była dosyć duża w porównaniu\
do wyniku pomiaru.
\end{document}

\documentclass[11pt]{article}

\usepackage[T1]{fontenc}
\usepackage[polish]{babel}
\usepackage[utf8]{inputenc}
\usepackage{lmodern}
\usepackage{amsfonts}
\usepackage{enumerate}
\usepackage{graphicx}
\usepackage{float}
\usepackage[margin=1in]{geometry}

\selectlanguage{polish}
\graphicspath{{./images/}}

\begin{document}
\title{Sprawozdanie 123}
\author{Jacek Gosztyła, Antoni Mleczko}
\maketitle
\section{Cel ćwiczenia} 
Celem ćwiczenia było zapoznanie się ze sposobem wyznaczania poziomej składowej magnetycznej\
magnetycznego pola ziemskiego przy użyciu przyrządu nazywanego busolą stycznych,\
a także zaobserwowanie podstawowych praw fizycznych wiążących prąd elektryczny z indukcją magnetyczną.
\section{Wstęp teoretyczny}

\section{Realizacja doświadczenia}
\subsection{Pomiary}
\subsection{Wykresy}
\section{Wnioski}

\end{document}

